\documentclass[10pt,a4paper]{article}


\usepackage[a4paper,bindingoffset=0.2in,%
left=1in,right=1in,top=1in,bottom=1in,%
footskip=.25in]{geometry}

\usepackage[latin1]{inputenc}
\usepackage{amsmath}
\usepackage{amsfonts}
\usepackage{amssymb}
\usepackage{makeidx}
\usepackage{graphicx}
\usepackage{float}
\usepackage{indentfirst}
\title{ARLISS 2016 Proposal}
\author{Damion Rosbrugh}

\begin{document}
\section{introduction}
Arliss is A Rocket Launch for International Student Satellites that happens annually in the month of September, in Black Rock Nevada. It is an international competition that Windward Community College (WCC) has attended every year since 2005.\paragraph{}Historically Windward Community College (WCC) has built a few different types of entries for the Arliss competition in Black Rock Nevada. One of the entries that has been used as recently as Arliss 2015 is the parasail design, whereby the unit would free fall for a given length of time and then deploy a parasail which will steer the unit, via servo control, to the target. This approach has many flaws which make it extremely difficult to achieve the goal of reaching the target. A major flaw in this design is that given a specific motor command to say "go right", it is difficult if not impossible to measure what the real world effect of such a command. In other words it is extraordinarily difficult to make predictions about the unit's future position based on it's current location, using physics. This has several ramifications on the methodology used to write the navigation algorithm. In addition, because unit lacks a powered descent, the navigation algorithm that we have been using depends critically on the angle between the unit's heading and the line that represents the distance to the target, as this value is changing rapidly, this causes a lot of noise in the data that the unit is using to determine its path. This means that the unit will have to make constant course corrections. In addition, the algorithm while well suited to the task of dealing with the complexities of a parasail descent is considerably complex by comparison to the algorithm being proposed this year. I am hopeing that the algorithm that I am proposing will be simpler yield a straight line path to the target. Because of it's tendency toward linear motions, a quadcopter design has the potential to yield the results we desire. For Arliss 2016, I am therefore proposing a quadcopter design which should mitigate many of the issues we have face, due of its natural linear motions and its ability to descend in a much more controlled fashion.

\section{The Competition}
In this section the rules and parameters for the competition are laid out. The biggest constraint on our design come from the size and shape of the rocket carrier. With dimensions of...... we have to ensure that our quad will fit. One rule for the competition states that we are required to log turn commands in order to be considered for the prize.

\section{Quadcopter Componants}
This section will outline the basic design and hardware needed for this project. The quadcopter being proposed has five main components in addition to perphierals that may be added time permitting. The build of this unit fall under the following six categories.
\begin{itemize}
	\item Structure
	\item Microcontroller
	\item Navigation
	\item Sensors
	\item Power
	\item Peripherals
\end{itemize}

\subsection{Structure}

The structure for this unit can be acquired by two means. We can buy a pre-built quadcopter frame and modify it to suit our needs, or we can 3-D print one to exact specifications. The latter option seems more appropriate since we have specialized needs as imposed by the physical constraints of the rocket carrier, as well as the resources to carry out.  

\subsection{Microcontroller}
The Pro-Micro-5v/16 MHz should be small enough and sufficient for our needs. It has a built voltage regulator, and micro-usb connector which will make in field code updates easier. 

\subsection{Navigation}

The parallax PAM-7Q gps module is small reasonably priced.


\subsection{Power System}
Graupner RC LiPo Battery 1S6P 3.7V 6000mAh
Next Level 20 Amp Multi-Rotor ESC with SimonK Firmware 
\subsection{Peripherals}
\begin{itemize}

\item TTL Serial JPEG Camera with NTSC Video

\item photo sensor for detecting when drone has left the rocket body
\item downward facing cam-Miniature TTL Serial JPEG Camera with NTSC Video


\item BMP180 Barometric Pressure/Temperature/Altitude Sensor- 5V ready
\end{itemize}

\begin{figure}[H]
\centering
\includegraphics[width=0.7\linewidth]{C:/Users/dhaead/Desktop/vtolcustom2}
\caption{Single propeller drone}
\label{fig:vtolcustom2}
\end{figure}
url{http://copter.ardupilot.com/wiki/singlecopter-and-coaxcopter/}



\pagebreak
\section{Algorithm}

Because the quadcopter will utilize a controlled descent, and has the capability to move in straight lines, the algorithm can be greatly simplified to simple vector algebra. In addition, because of this linear motion, we are able to do tests and calibrate to determine what exactly a "go right" command means in the real world. Furthermore, because of our ability to measure this and make predictions about future positions as a result of a given motor command, we should be able to increase our unit's position resolution to better than what a factory GPS unit can provide (10 meters). We can do this by using a Kalman filter which, minimizes the mean square error between our state prediction and our sensor estimation. The state prediction can be worked out based on the motor command in conjunction with the physics of the system. The sensor estimation is simply the GPS coordinates received from the GPS sensor.\paragraph{}
The navagation algorithm will be programmed using the following logic. Given two coordinates (target coordinates and quadcopter position) in a three dimensional vector space, where each set of coordinates is given by gps location and altitude, we can calculate the vector that connects the two. Recognizing that this vector is a linear combination of unit vectors multiplied by their respective magnitudes, we can program the quadcopter to decompose this vector in a set of vector componants. With the componants know, and assuming the relationship between voltage applied and propeller rpms is linear, we can create a voltage profile for unit movement in $\Re^3$. To put it more concretely, since translational movement of the quad is achieved by applying more voltage to the motors furthest from the direction of travel than the motors closest, we can send a voltage profile to the props, to be applied for a specific length of time, and measure the distance traveled. With some calibration we should be able to easily determine the proper profile for traveling say 1 meter. This gives us the oppurtunity to determine unit translation profiles for the quad, and since the relationship between voltage and rpm is linear and we need a linear combination of vector componants to translate to the target, we should be able to follow a straight line path directly to the target by adding the appropriate combinations of voltage profiles scaled by time. This algorithm should be tested extensively throughout the year. With testing, the findings of the tests should be analyzed and interpreted such that a decision can be made to modify the algorithm or code in the appropriate way. \paragraph{}If time permits, and assuming this algorithm is programmable and works, I would like to look into applying a Kalman filter to the algorithm to increase position resolution beyond the capability of the normal off the shelf GPS module.
In addition, a camera could be fitted to the unit which could be used to look for the target in close proximity and when the camera sees it, the algorithm will head toward that and no longer depend on the GPS. There is also the possibility of increasing the devices gps resolution by use of a Kalman filter which uses statistical methods to minimize the mean square error in the measurements. 

\section{code}
The code for this project will be a modified version of the code used for ARLISS 2015 which was provided by Helen Rapozo. The code will be tested along side of the algorithm and debugged over the course of the year. The code must contain a section for logging turn commands. 


\section{Calculating Motor Thrust}
Based on current understanding, we need a thrust to weight ratio of about 2:1 or better. Below is an equation I found to determine static thrust. The current best estimate of the weight without the body is about 291 grams. This is a very rough approximation and should be made more accurate before ordering parts.\\

\begin{equation} T = [ (eta*P)^2 * 2 * \pi *R^2 * \rho ]^{0.3333}\end{equation}\\

Where T is the thrust in Newtons, $\rho$ is the air density, which about $1.22 kg/m^3$
,R is the prop radius (in meters), eta is the prop hover efficiency, 0.7-0.8 is typical for unstalled low-pitch props. $P$ is the shaft power, and is given by\\
\begin{equation}P = V * I * eff \end{equation}

Where $V$ is the voltage, $I$ is the current and $eff$ is the motor efficiency in watts.\\



In this equation i'm not sure what to use for eta (i was thinking 0.75). I also don't know what I should use for current, voltage, motor efficiency and propeller radius. Should i just use the max current and voltage that the motor can take which is 17A and 11V respectively. Should I just assume the motor is around 75\% efficient? Will the prop radius just be half the length of the propeller, in my case that will be 5 in (0.127 m).

If i use the values i assumed above i get,

$T = [(0.75*11*17*0.75)^2 * 2 * \pi * 0.127 *0.127 * 1.22]^{0.3333} (0.127 is the radius in m)$
$T ~= 11.1 N$

\section{Testing and Calibration}
Several tests will need to occur throughout the year to ensure everything is working as it should. The following are a list of testing schemes which must be conducted in order for this project to succeed.


Determine voltage profile for a one meter translation in all six directions

\section{Budget}

Based on currently available information, the estimated total budget for the quadcopter is about \textdollar319.10 and the total estimated cost for the complete project with airfair, a spare quadcopter, and a hotel room is about \textdollar3057.30. A complete budget breakdown can be found in the document Arliss 2016 Budget.


\section{References}
url{http://aeroquad.com/showthread.php?1048-Calculating-motor-thrust}



\end{document}
